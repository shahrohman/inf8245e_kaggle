\documentclass[11pt,a4paper]{article}
\usepackage[T1]{fontenc}     
\usepackage[utf8]{inputenc}  % Accents codés dans la fonte
\usepackage{natbib}
\usepackage[french]{babel}  % Les traductions françaises
\usepackage{numprint}        % \numprint(9,36) pour utilisation de la virgule comme séparateur décimal
\usepackage{caption}
\usepackage{amsmath}         % Les maths de base

\usepackage[svgnames]{xcolor}% Pour les besoins de PythonTeX
\usepackage{minted}          % Pour présenter du code quelconque (C, java,...)
\usepackage[margin=2.5cm]{geometry}

%\usepackage{tgpagella}       % Pour changer un peu les fontes
%\usepackage{tgadventor}
%\usepackage{inconsolata}
\usepackage{lmodern}


% %\usepackage{minted}
% \usepackage{pythontex}       % Utilisation de PythonTeX

%\usepackage{sagetex}        % Utilisation de SageMath

\usepackage{graphicx}        % Gestion des inclusions graphiques
\usepackage{subcaption}
\usepackage{float}
\usepackage{wrapfig}
\usepackage{titlepic}
%\usepackage[frenchb]{babel} \usepackage{hyperref}

\usepackage{comment} % Pour pouvoir commenter facilement

% \usepackage{tikz}            % Si on veut présenter le code Python
% \usepackage[framemethod=TikZ]{mdframed}
% % Un environnement pour faire joli pour présenter le code Python
% \usepackage[framed,numbered,autolinebreaks,useliterate]{mcode}
% \usepackage{hyperref}

\usepackage{fancyhdr}
\usepackage{lastpage}
\usepackage{perpage}
\usepackage[bottom]{footmisc}
\usepackage{lipsum}
\usepackage{colortbl}
\usepackage{hyperref}
\usepackage{caption}

% Couleurs
\definecolor{rouge_doux}{RGB}{241, 148, 138}
\definecolor{rouge_moyen}{RGB}{236, 112, 99}
\definecolor{gris}{RGB}{224, 224, 224}

% Tableaux
\newcolumntype{L}{>{\centering\arraybackslash}m{3cm}}

\MakePerPage{footnote} %Note bas de pages

% en-têtes et pieds de page
\renewcommand{\headrulewidth}{1pt} 
\renewcommand{\footrulewidth}{1pt} 
\setlength{\headheight}{42pt}
\setlength{\headsep}{15pt}

\lhead{\textsc{???} Andréa - XXXXXXX\\textsc{TREMBLY} Timothée - XXXXXXX\\\textsc{ROHMAN} Shahriar - 2164342}
\chead{}
\rhead{\includegraphics[scale=0.12]{img/poly-logo.png}}
\lfoot{\textsc{Polytechnique Montréal}}
\cfoot{Page \thepage/\pageref{LastPage}}
\rfoot{\today}

\newenvironment{code}{%
\begin{mdframed}[linecolor=Green,innerrightmargin=30pt,innerleftmargin=30pt,
backgroundcolor=Black!5,
skipabove=10pt,skipbelow=10pt,roundcorner=5pt,
splitbottomskip=6pt,splittopskip=12pt]
}{%
\end{mdframed}
}

% Un raccourci pour composer les unités en caractères droits
\newcommand{\U}[1]{~\mathrm{#1}}

% Présentation de l'abstract pour la problématique
% \usepackage[runin]{abstract}

% Un environnement pour la problématique
% \newenvironment{problematique}{
% \renewcommand{\abstractname}{Problématique}
% \begin{abstract}
%}{
%\paragraph{}
%
%\end{abstract}
%}

%régler l'espacement entre les lignes
\newcommand{\HRule}{\rule{\linewidth}{0.5mm}}

% Titre et auteurs du document
\begin{document}

\begin{center}
    % Upper part of the page. The '~' is needed because only works if a paragraph has started.
    
    \includegraphics[width=0.76\textwidth]{img/poly-logo.png}
    ~\\[1cm]
    
    \textsc{\LARGE Département de génie informatique \\\textbf{École Polytechnique de Montréal}}\\[1.5cm]
    
    \textsc{\Large INF8245E : Machine Learning}\\[0.5cm]
    
    % Title
    \HRule \\[0.5cm]
    
    {\huge \bfseries Kaggle project} \\[0.5cm]
    
    \HRule \\[1.5cm]
    
    % Author and supervisor
    \begin{minipage}{0.4\textwidth}
    \begin{flushleft} \large
    \textbf{\LARGE Auteurs:}\\
    \Large \textsc{Gourion} Andréa\\
    {\large Matricule XXXXXXX}\\
    \Large \textsc{Trembly} Timothée\\
    {\large Matricule XXXXXXX}\\
    \Large  \textsc{Rohman} Shahriar\\
    {\large Matricule 2164342}
    \end{flushleft}
    \end{minipage}
    \begin{minipage}{0.4\textwidth}
    \begin{flushright} \large
    \end{flushright}
    \end{minipage}
    
    \vspace{2cm}
    
    \textsc{\large \today}
    
    \thispagestyle{empty}
    
\end{center}

% % Pour afficher la numérotation à droite et non au centre
\pagestyle{fancy}

%\tableofcontents

\newpage

\section{Introduction}

\section{Pre-processing}

\subsection{Problématique}

\paragraph{} Données manquantes, données en chaînes de caractères, données aberrantes, ect...

\subsection{Data cleaning}

\paragraph{} On a enlevé les NaN

\subsection{Data transformation}

\paragraph{} On a converti les string en int à partir de dictionnaires

\section{Modélisation}

\subsection{Comparaison des modèles}

\paragraph{} On a comparé les modèles suivants : SVM, KNN, Random Forest, Decision Tree, Logistic Regression, Naive Bayes, ect...

\subsection{Modèle choisi}

\paragraph{} On a choisi le modèle XX qui donne les résultats suivants :


\end{document}